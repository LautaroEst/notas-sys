\documentclass[12pt]{article}
\usepackage[utf8]{inputenc}
\usepackage[spanish,es-tabla]{babel}
\usepackage[autostyle=false,style=american]{csquotes}
\MakeOuterQuote{"}


% Matemática:
\usepackage{marvosym}
\usepackage{amsmath}
\usepackage{amsfonts}
\usepackage{amsthm}
\usepackage{amssymb}
\usepackage{physics}

% Gráficos e imágenes:
\usepackage{graphicx}
\graphicspath{{images/}}
\usepackage{caption}
\usepackage{subcaption}

% Formato:
\usepackage[left=2.3cm,right=2.3cm,top=2cm,bottom=2cm]{geometry}
\usepackage{xcolor}
\usepackage[outline]{contour}

% \usepackage[toc, titletoc, page]{appendix}

% Tikz:
\usepackage{tikz}
\usetikzlibrary{arrows,shapes,positioning,babel,angles,quotes,bending,3d}
\usepackage{pgfplots}
\pgfplotsset{compat=1.18}
\usepackage{array}
\usepackage{tabularray}
\contourlength{1.0pt}
% \tikzset{>=latex}
\allowdisplaybreaks

% Referencias:
\usepackage{biblatex}
\addbibresource{01_señales/references.bib}
\addbibresource{02_sistemas/references.bib}
\addbibresource{03_series_fourier/references.bib}
\addbibresource{04_transformada_fourier/references.bib}
\addbibresource{05_muestreo/references.bib}
\addbibresource{06_dft/references.bib}
\addbibresource{07_espectrograma/references.bib}
\addbibresource{08_filtros/references.bib}

% Outline:
\usepackage[bookmarks]{hyperref}



%% COMANDOS %%

% Símbolos
\newcommand{\Keyboardsym}{\Keyboard \hspace{.3em}}
\newcommand{\Realpart}[1]{\mathfrak{Re}\left\{#1\right\}}
\newcommand{\Impart}[1]{\mathfrak{Im}\left\{#1\right\}}
\newcommand{\R}{\mathbb{R}}
\newcommand{\C}{\mathbb{C}}
\newcommand{\Z}{\mathbb{Z}}
\colorlet{myblue}{blue!65!black}
\colorlet{mydarkblue}{blue!50!black}
\colorlet{myred}{red!65!black}
\colorlet{mydarkred}{red!40!black}
\colorlet{veccol}{green!70!black}
\colorlet{vcol}{green!70!black}
\colorlet{xcol}{blue!85!black}
\tikzstyle{vector}=[->,very thick,xcol,line cap=round]
\tikzstyle{xline}=[myblue,very thick]
\tikzstyle{yzp}=[canvas is zy plane at x=0]
\tikzstyle{xzp}=[canvas is xz plane at y=0]
\tikzstyle{xyp}=[canvas is xy plane at z=0]
\def\tick#1#2{\draw[thick] (#1) ++ (#2:0.12) --++ (#2-180:0.24)}
\def\N{100}

% Teoremas, ejercicios, contadores, etc
\newcounter{ejercicioi}
\newcommand{\ejercicios}{\setcounter{ejercicioi}{0}}
\newenvironment{ejercicio}{
    \incisos
    \vspace{0.5cm}
    \hrule
    \vspace{0.3cm}
    \setcounter{equation}{0}
    \refstepcounter{ejercicioi}
    \noindent
    {\bf \arabic{ejercicioi}.} 
}{}
\newcounter{incisoi}[ejercicioi]
\newcommand{\incisos}{\setcounter{incisoi}{0}}
\newcommand{\inciso}{
    \vspace{0.2cm}
    \refstepcounter{incisoi} \noindent (\textbf{\alph{incisoi}})~
}
\theoremstyle{definition}
\newtheorem{definition}{Definición}[section]
\theoremstyle{definition}
\newtheorem{theorem}{Teorema}[section]
\theoremstyle{definition}
\newtheorem{lemma}{Lemma}[section]
\theoremstyle{definition}
\newtheorem{example}{Ejemplo}[section]
\theoremstyle{definition}
\newtheorem{property}{Propiedad}[section]
\renewcommand*{\proofname}{Demostración}
% \renewcommand\qedsymbol{$\blacksquare$}

% Title page
\newcommand{\mytitle}[1]{
    \begin{titlepage}
        \centering
        \vspace*{10em}
        \LARGE\textsc{Notas de Señales y Sistemas} \\[5em]
        \Large\textsc{#1} \\
        \vspace*{10em}
        \newpage
    \end{titlepage}       
    \tableofcontents
    \newpage
}

% Unidad temática
\newcommand{\unidadtem}{0}

\begin{document}

\ifcase%
\unidadtem{} {
        \mytitle{Señales de Tiempo Continuo y Discreto}


\section{Señales de Tiempo Continuo}

\section{Señales de Tiempo Discreto}


        \newpage
        \mytitle{Caracterización de Sistemas}
        \newpage
        \mytitle{Caracterización de Sistemas}
        \newpage
        \mytitle{Caracterización de Sistemas}
        \newpage
        \mytitle{Muestreo e Interpolación}
        \newpage
        \mytitle{Transformada Discreta de Fourier}
        \newpage
        \mytitle{Transformada de Laplace y Z}
        \newpage
        \mytitle{Caracterización de Sistemas}
    }
    \or \mytitle{Señales de Tiempo Continuo y Discreto}


\section{Señales de Tiempo Continuo}

\section{Señales de Tiempo Discreto}


    \or \mytitle{Caracterización de Sistemas}
    \or \mytitle{Caracterización de Sistemas}
    \or \mytitle{Caracterización de Sistemas}
    \or \mytitle{Muestreo e Interpolación}
    \or \mytitle{Transformada Discreta de Fourier}
    \or \mytitle{Transformada de Laplace y Z}
    \or \mytitle{Caracterización de Sistemas}
    \else UNIDAD TEMÁTICA NO SELECCIONADA!!
\fi

\newpage

\appendix


\section{Algunas definiciones matemáticas}


\begin{definition}[Intervalo Real]
    \label{def:real_interval}
    Un intervalo real es un conjunto $\mathcal{I} \subseteq \R$ tal que, para cualesquier $a,b \in \mathcal{I}$ y $t \in \R$ se cumple que si $a \leq t \leq b$ entonces $t \in \mathcal{I}$.
\end{definition}



\printbibliography
\addcontentsline{toc}{section}{Bibliografía}

\end{document}