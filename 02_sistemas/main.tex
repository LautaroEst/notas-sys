\mytitle{Caracterización de Sistemas}

Introducción 

\section{Clasificación de sistemas}

Dar la clasificación de sistemas en general, sin distinguir entre discreto y continuo. Ejemplos de sistemas. Ejemplos de sistemas en cascada.

\begin{definition}[Linearidad]
    Se dice que un sistema $\mathcal{T}$ es lineal si y sólo si se cumple que $\mathcal{T}(a_1x_1(t) + a_2x_2(t)) = a_1\mathcal{T}(x_1(t)) + a_2\mathcal{T}(x_2(t))$ para todo $t \in \mathcal{I}$ y $a_1,a_2 \in \mathcal{R}$.    
\end{definition}

\begin{example}
    Ejemplo corto de sistema lineal.
\end{example}

\begin{example}
    Ejemplo corto de sistema no lineal.
\end{example}

\begin{example}
    Ejemplo de reconstrucción de la entrada a partir de la salida de un sistema lineal.
\end{example}

\begin{definition}[Invarianza en el tiempo]
    Se dice que un sistema $\mathcal{T}$ es invariante en el tiempo si y sólo si la salida es la misma cuando se efectúa la operación desplazamiento antes o después de pasar por el sistema. Es decir, si $\mathcal{T}(\mathcal{D}_{\tau}(x)) = \mathcal{D}_{\tau}(\mathcal{T}(x))$.
\end{definition}

\begin{example}
    Ejemplo corto de sistema invariante en el tiempo.
\end{example}

\begin{example}
    Ejemplo corto de sistema no invariante en el tiempo.
\end{example}

\begin{example}
    Ejemplo de reconstrucción de la entrada a partir de la salida de un sistema invariante en el tiempo.
\end{example}

\begin{definition}[Memoria]
    Se dice que un sistema $\mathcal{T}$ tiene memoria si y sólo si la salida en el instante $t$, $y(t)$, depende únicamente del valor de la entrada en ese momento. Es decir, si existe una función $f: \mathcal{Y} \rightarrow \mathcal{Y} : f(x(t)) = \mathcal{T}(x)(t) \; \forall t \in \mathcal{I}$.
\end{definition}

\begin{definition}[Causalidad]
    Se dice que un sistema es causal si y sólo si su salida en el instante $t$, $y(t)$, depende únicamente del valor de la entrada en el instante $t$ o en instantes anteriores. Formalmente, un sistema es causal si y sólo si, para todo $t_0 \in \mathcal{I}$ se cumple que si $x_1(t) = x_2(t)$ para todo $t < t_0$, entonces $y_1(t) = y_2(t)$ para todo $t < t_0$.
\end{definition}

Es decir, que si dos entradas $x_1(t),x_2(t)$ son iguales hasta un instante $t_0$ y el sistema es causal, entonces las correspondientes salidas también serán iguales entre sí para todo hasta el instante $t_0$. Y esto es verdad para cualquier instante $t_0$.

\begin{definition}[Invertibilidad]
    Se dice que un sistema $\mathcal{T}$ es inversible si y sólo si existe una función $\mathcal{T}^{-1}: \mathcal{Y} \rightarrow \mathcal{X} : g(y(t)) = x(t) \; \forall t \in \mathcal{I}$.
\end{definition}

\begin{definition}[Estabilidad BIBO]
    Se dice que un sistema $\mathcal{T}$ es estable BIBO (o, para abreviar, "estable") si y sólo si entradas acotadas generan salidas acotadas. Es decir, si y sólo si se cumple que: si existe una constante $B_1 > 0$ tal que $|x(t)| \leq B_1 \forall t \in \mathcal{I}_x$, entonces existe una constante $B_2 > 0$ tal que $|y(t)| \leq B_2 \forall t \in \mathcal{I}_y$.
\end{definition}

\section{Sistemas LTI}

\begin{definition}[Sistema LTI]
    Un sistema es LTI si y sólo si es lineal e invariante en el tiempo.
\end{definition}

\begin{example}
    Ejemplo corto de sistema LTI.
\end{example}

\begin{example}
    Reconstrucción de la entrada a partir de la salida de un sistema LTI.
\end{example}

\subsection*{Propiedades de los sistemas LTI}

\begin{property}[Reconstrucción de la salida]
    Dado un conjunto de entradas $\mathcal{G} = \{x_1(t),\ldots,x_n(t)\}$, puedo obtener la salida de cualquier señal que sea combinación lineal de $\mathcal{G}$ y del que se obtiene de desplazar temporalmente cada una de las señales de $\mathcal{G}$.
\end{property}

Como caso particular de lo anterior se tiene lo siguiente:

\begin{property}[Convolución]
    Si se conoce la respuesta al impulso del sistema, es decir, salida para $\delta(t)$, es posible obtener cualquier salida del sistema a partir de la convolución de la entrada con la respuesta al impulso del sistema.
\end{property}

\begin{property}[Conexión de sistemas LTI]
    La conexión en cascada, la suma y la conmutación de sistemas LTI también es un sistema LTI. 
\end{property}

\begin{property}[Respuesta a exponenciales en sistemas de tiempo continuo]
    La respuesta de un sistema LTI de tiempo continuo y respuesta al impulso $h(t)$ a una señal exponencial $e^{st}, s\in \C$ es una señal exponencial escalada por un factor que depende únicamente de $s$. Por lo tanto, $e^{st}$ es un autovector del sistema, y su correspondiente autovalor es:
    \begin{equation*}
        H(s) := \int_{-\infty}^{\infty} h(t)e^{-st}dt
    \end{equation*}
\end{property}

\begin{property}[Respuesta a exponenciales en sistemas de tiempo discreto]
    La respuesta de un sistema LTI de tiempo discreto y respuesta al impulso $h(n)$ a una señal exponencial $z^{n}, z \in \C$ es una señal exponencial escalada por un factor que depende únicamente de $z$. Por lo tanto, $z^{n}$ es un autovector del sistema, y su correspondiente autovalor es:
    \begin{equation*}
        H(z) := \sum_{n=-\infty}^{\infty} h(n)z^{-n}
    \end{equation*}
\end{property}

\begin{property}[Respuesta a señales periódicas]
    La respuesta de un sistema LTI a una señal periódica con desarrollo en serie de fourier 
    \begin{equation*}
        x(t) = \sum_{k=-\infty}^{\infty} a_k e^{-j\frac{2\pi}{T}kt}
    \end{equation*}
    es una señal periódica con desarrollo en serie de fourier 
    \begin{equation*}
        y(t) = \sum_{k=-\infty}^{\infty} b_k e^{-j\frac{2\pi}{T}kt}    
    \end{equation*}
    donde $b_k = H\left(e^{-j\frac{2\pi}{T}k}\right)a_k$.
\end{property}